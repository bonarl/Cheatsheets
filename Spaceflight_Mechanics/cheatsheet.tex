\documentclass[table,cmyk]{article}
\usepackage[a4paper,margin=0.5cm,portrait]{geometry}
\usepackage{longtable,array,calc}
\usepackage{xcolor}
\usepackage{mathtools}
\usepackage{braket}
\makeatletter
\newcommand\ratio[2]{\strip@pt\dimexpr#1pt/#2\relax}
\newcolumntype{A}[2]
{
        >{\begin{minipage}[t]{#2\linewidth-2\tabcolsep-#1\arrayrulewidth}%
        \vspace{\tabcolsep}}%
        c%
        <{\vspace{\tabcolsep}\end{minipage}}%   
}
\makeatother

\pagestyle{empty}

\arrayrulewidth=1pt
\tabcolsep=10pt
\arrayrulecolor{blue}


\begin{document}
\begin{longtable}
{
    |A{1.5}{\ratio{50}{100}}% 30%
    |A{1}{\ratio{50}{100}}% 30%
    |%
}\hline
\section*{Dynamics}
\subsection*{Relative Motion for Rotating Frame}
\[\underline{v} = \frac{d \underline{r}}{d t} = \underline{v}_{o'} + \underline{v}_r + \underline{w} \times \underline{r}\]
\[ \frac{d\underline{v}}{dt} = \underline{a}_r + 2\underline{\omega}\times\underline{v}_r + \frac{d\underline{\omega}}{dt}\times\underline{r}+\underline{a}_{o'}+\underline{\omega}\times(\underline{\omega}\times\underline{r})\]
Polar Coordinates:
\[\underline{v} = \dot{r}\underline{e}_r + r\dot{\theta}\underline{e}_{\theta}\]
\[\underline{a} = (\ddot{r}-r\dot{\theta}^2)\underline{e}_r + (r\ddot{\theta} + 2\dot{r}\dot{\theta})\underline{e}_{\theta}\]
Binet's Formula:

For $\underline{a} \times \underline{r} = 0$ we have $a_{\theta} = 0 \implies r^2 \dot{\theta} = c$ which gives

\[a_r = -\frac{c^2}{r^2}\left[\frac{d^2}{d\theta^2}\left(\frac{1}{r}\right)+\frac{1}{r}\right]\]

\vspace{0.5cm}
\noindent\rule{9cm}{0.4pt}
\subsection*{Gravity}
\[m\underline{a} = -\frac{GmM}{r^2}\underline{\hat{r}} = -\frac{m\mu}{r^2}\underline{\hat{r}}\]
Total Mechanical Energy:
\[E = \frac{1}{2}mv^2 - \frac{m\mu}{r}\]
Angular Momentum:
\[\underline{L} = \underline{r}\times m \underline{v} \]
\[|\underline{L}| = mr^2\dot{\theta} \implies \text{conserved}\]

\vspace{0.5cm}
\noindent\rule{9cm}{0.4pt}
\section*{Keplerian Motion}
\subsection*{Two-Body Dynamics}
Gravitational Forces:
\[m_i\ddot{\underline{r}}_i=\pm \frac{Gm_im_j}{r^2}\underline{\hat{r}}\]
Centre-of-Mass:
\[\underline{r}_C = \frac{m_1\underline{r}_1 + m_2\underline{r}_2}{m_1+m_2}\]
Sum of forces acting on centre of mass $= 0 \implies \underline{r}_C = \underline{r}_0 + \underline{r}'t$
Approximate $\mu \approx Gm_1$ gives:
\[\ddot{\underline{r}} = -\frac{\mu}{r^2}\underline{\hat{r}}\]
Total energy is conserved, take $\ddot{\underline{r}}\cdot\underline{r}$ to prove $(d/dt) E = 0$, since:
\[\frac{d}{dt}\left(\frac{1}{2}v^2\right) = \underline{\ddot{r}}\cdot\underline{\dot{r}} =  \frac{d}{dt}\left(\frac{G(m_1+m_2)}{r}\right)\]
&
\subsection*{Equations of Motion}
Use chain rule and $r^2\dot{\theta} = l$ to get:
\[\frac{dr}{dt} = \frac{h}{r2}\frac{dr}{d\theta}\]

Change variable to $u = 1/r$:
\[\frac{d}{dt} = -h \frac{du}{d\theta}\]

Gives:
\[\frac{d^2r}{dt^2} = -h^2u^2\frac{d^2u}{d\theta^2}\text{\hspace{1cm}and\hspace{1cm}} -r\dot{\theta}^2 = -h^2u^3\]
Equations of motion in polar coordinates are:
\[\ddot{r} - r\dot{\theta}^2 = -\frac{\mu}{r^2}\text{\hspace{1cm}and\hspace{1cm}}r\ddot{\theta}+2\dot{r}\dot{\theta} = \frac{1}{r}\frac{d}{dt}(r^2\dot{\theta})=0\]
Transforming for $u$:
\[-l^2u^2\frac{d^2u}{d\theta^2}-l^2u^3 = -\mu u^2\]
\[\frac{d^2u}{d\theta^2}+u = \frac{\mu}{l^2}\]
Finally change to $z = u-\mu/l^2$ to get:
\[\frac{d^2z}{d\theta^2}+z = 0 \implies z = A\cos(\theta-\theta_0)\]
Which gives polar equation of conic section for $r$:
\[r(\theta) = \frac{l^2/\mu}{1+\frac{Al^2}{\mu}\cos(\theta)}\]
The energy equation is :
\[E = \frac{1}{2}m(\dot{r}^2+r^2\dot{\theta}^2)-U(r)\]
Which can be solved by integrating w.r.t $r$, changing variable from $t$ to $\theta$, and substituting $U(r) = m\mu/r$. This gives:
\[r = \frac{p}{\pm1+e\cos(\theta-\theta_0)}\]
where
\[p = \frac{c^2}{|\mu|}=a(1-e^2)\text{\hspace{1cm}and\hspace{1cm}}e = \sqrt{\frac{2Ec^2}{m\mu^2}+1}\]
so r describes an ellipse with semimajor axis $a$ and eccentricity $e$, pericentre at $\theta=0$ and apocentre at $\theta=\pi$:
\[r = \frac{a(1-e^2)}{1+e\cos(\theta-\theta_0)}\]



\vspace{2cm}


\tabularnewline\hline
\section*{Orbital Parameters}
\[a=\frac{r_p+r_a}{2}\text{\hspace{1cm}and\hspace{1cm}}e=\frac{r_a-r_p}{r_a+r_p}\]
so
\[r_p=a(1-e)\text{\hspace{1cm}and\hspace{1cm}}r_a=a(1+e)\]
Comparing equations for $r$:
\[l^2 = \mu a(1-e^2)\]
\[v_p = \sqrt{\frac{\mu(1+e)}{a(1-e)}}\text{\hspace{1cm}and\hspace{1cm}}v_a = \sqrt{\frac{\mu(1-e)}{a(1+e)}}\]
Conservation of energy:
\[E = -\frac{\mu}{2a} \implies v^2 = \mu\left(\frac{2}{r}-\frac{1}{a}\right)\]
\subsection*{Kepler's Laws}
1st Law: Orbits are elliptical with attracting body at focus of ellipse
\newline
\newline
2nd Law: Radial vector sweeps out equal areas in equal time

\[\implies T = 2\pi\sqrt{\frac{a^3}{\mu}} \]

3rd Law: Orbital period squared $\propto$ semi-major axis cubed:
\subsection*{Kepler's Equation}
Mean motion
\[n=\sqrt{\frac{\mu}{a^3}}\]
Mean anomoly
\[M=n(t-\tau)\]
Kepler's equation is
\[M=E-e\sin E\]
and use trig identity to get $\theta$ from $E$
\[\tan \left(\frac{\theta}{2}\right) = \sqrt{\frac{1+e}{1-e}}\tan \left(\frac{E}{2}\right)\]
This must be solved iteratively, start with guess $E_0 = M$, then find correction
\[\Delta E = \frac{M-E_0+e\sin E_0}{1-e\cos E_0}\]
Then repeat, using $E_i = E_{i-1}+\Delta E_{i-1}$ until $\Delta E_i \approx 0$

\subsection*{Position and Velocity from Orbital Elements}
From orbit equation
\[\dot{r} = \frac{a(1-e^2)}{(1+e\cos \theta)^2} e\sin \theta \dot{\theta} = r\frac{e\sin \theta}{1+e\cos \theta} \dot{\theta}\]
Flight path angle
\[\tan \gamma = \frac{v_r}{v_\theta}=\frac{\dot{r}}{r\dot{\theta}}=\frac{e\sin \theta}{1+e \cos \theta}\]
&
To get in plane position and velocity at time $t$, solve Kepler's equation for $\tau$ time of passage from pericentre, then compute orbit radius and thus cartesian positions.
\[x = r\cos \theta \text{\hspace{1cm}and\hspace{1cm}} y = r\sin \theta\]
Velocity components:
\[v_r = \frac{r^2e\sin \theta}{p}\dot{\theta}\text{\hspace{1cm}and\hspace{1cm}}v_\theta = \frac{l}{r} = r\dot{\theta}\]
Gives cartesian velocities:
\[\dot{x} = -\sqrt{\frac{\mu}{p}}\sin \theta \text{\hspace{1cm}and\hspace{1cm}}\dot{y} = \sqrt{\frac{\mu}{p}}(e+\cos \theta)\]
\noindent\rule{9cm}{0.4pt}
\section*{Orbit Transfers}
\subsection*{Hohmann Transfer}
Minimum energy transfer between two circular, coplanar orbits is an ellipse with semimajor axis $a = \frac{1}{2}r_1 + r_2$. The change in velocities required are the difference between the velocities of the transfer ellipse at pericentre and apocentre, and the velocities of the circular orbits at $r_1$ and $r_2$
\[\Delta v_1 = \sqrt{\frac{2\mu}{r_1}-\frac{2\mu}{r_1+r_2}}-\sqrt{\frac{\mu}{r_1}}\]
\[\Delta v_2 = \sqrt{\frac{\mu}{r_2}}-\sqrt{\frac{2\mu}{r_2}-\frac{2\mu}{r_1+r_2}}\]
\subsection*{Parabolic Transfer}
For a parabolic transfer where the orbiting body escapes circular orbit at $r_1$ to infinity, and is then captured from infinity into orbit at $r_2$
\[\Delta v_1 = (\sqrt{2}-1)\sqrt{\frac{\mu}{r_1}}\]
\[\Delta v_2 = (\sqrt{2}-1)\sqrt{\frac{\mu}{r_2}}\]
\subsection*{Single Impulse Manoeuvres}
Raise apocentre:
\[\Delta v = \sqrt{-\frac{\mu}{a_2}+\frac{2\mu}{r_p}}-\sqrt{-\frac{\mu}{a_1}+\frac{2\mu}{r_p}}\]
Change inclination:
\[\Delta v = \sqrt{v_1^2+v_2^2-2v_1v_2\cos i}\]
\tabularnewline\hline
\end{longtable}
\end{document}