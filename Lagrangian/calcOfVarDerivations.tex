\documentclass[table,cmyk]{article}
\usepackage[a4paper,margin=0.5cm,portrait]{geometry}
\usepackage{longtable,array,calc}
\usepackage{xcolor}
\usepackage{mathtools}
\usepackage{braket}
\makeatletter
\newcommand\ratio[2]{\strip@pt\dimexpr#1pt/#2\relax}
\newcolumntype{A}[2]
{
        >{\begin{minipage}[t]{#2\linewidth-2\tabcolsep-#1\arrayrulewidth}%
        \vspace{\tabcolsep}}%
        c%
        <{\vspace{\tabcolsep}\end{minipage}}%   
}
\makeatother

\pagestyle{empty}

\arrayrulewidth=1pt
\tabcolsep=10pt
\arrayrulecolor{blue}


\begin{document}
\begin{longtable}
{
    |A{1.5}{\ratio{50}{100}}% 30%
    |A{1}{\ratio{50}{100}}% 30%
    |%
}\hline
\section*{Functional Minimisation}
  Can derive lagrange's equations from minimising the functional, $F[y]$.
  &
\section*{Beltrami Identity}
\tabularnewline\hline

\end{longtable}
\end{document}