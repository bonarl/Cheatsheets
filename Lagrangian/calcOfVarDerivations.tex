\documentclass[table,cmyk]{article}
\usepackage[a4paper,margin=0.5cm,portrait]{geometry}
\usepackage{longtable,array,calc}
\usepackage{xcolor}
\usepackage{mathtools}
\usepackage{braket}
\makeatletter
\newcommand\ratio[2]{\strip@pt\dimexpr#1pt/#2\relax}
\newcolumntype{A}[2]
{
        >{\begin{minipage}[t]{#2\linewidth-2\tabcolsep-#1\arrayrulewidth}%
        \vspace{\tabcolsep}}%
        c%
        <{\vspace{\tabcolsep}\end{minipage}}%   
}
\makeatother

\pagestyle{empty}

\arrayrulewidth=1pt
\tabcolsep=10pt
\arrayrulecolor{blue}


\begin{document}
\begin{longtable}
{
    |A{1.5}{\ratio{50}{100}}% 30%
    |A{1}{\ratio{50}{100}}% 30%
    |%
}\hline
\section*{Functional Minimisation}
  Can derive Lagrange's equation from minimising the functional:
  \begin{displaymath}
   J[y]=\int_a^bF(\{y(x)\}, \{\dot y(x)\}, x)dx
  \end{displaymath}
  Let $y(x)\rightarrow y(x)+\epsilon\eta(x)$, $\dot y(x)\rightarrow \dot y(x) +
  \epsilon \dot \eta(x)$ with $\eta(a)=\eta(b)=0$

  $F$ is at a minimum with $\epsilon=0$, we therefore have:
  \begin{displaymath}
   \left \frac{dJ}{d\epsilon} \right |_{\epsilon=0}=0
  \end{displaymath}
  \begin{displaymath}
   \frac{dJ}{d\epsilon}=\int_a^b\frac{dF}{d\epsilon} dx=\int_a^b\left(
     \frac{\partial F}{\partial y}\frac{dy}{d\epsilon}+\frac{\partial
       F}{\partial \dot y}\frac{d\dot y}{d\epsilon} \right)dx
  \end{displaymath}
  Using:
  \begin{displaymath}
    \frac{dy}{d\epsilon}=\eta, \frac{d\dot y}{d\epsilon}=\dot\eta
  \end{displaymath}
  We have:
  \begin{displaymath}
    0=\int_a^b\left( \frac{\partial F}{\partial y}\eta + \frac{\partial
        F}{\partial \dot y}\dot \eta \right)dx
  \end{displaymath}
  Using integration by parts wrt. $x$ on $\frac{\partial F}{\partial \dot y}\dot
  \eta$ we get:
  \begin{displaymath}
   0 = \int_a^b \left( \frac{\partial F}{\partial y}\eta -
     \eta\frac{d}{dx}\frac{\partial F}{\partial \dot y}  \right) dx + \left[
     \frac{\partial F}{\partial \dot y}\eta \right]_a^b
  \end{displaymath}
  Which becomes (using $\eta(a)=\eta(b)=0$)
  \begin{displaymath}
   0 = \int_a^b\eta \left( \frac{\partial F}{\partial y} -
     \frac{d}{dx}\frac{\partial F}{\partial \dot y} \right) dx
  \end{displaymath}
  With $\eta$ being any general small displacement, only having requirements at
  points $a, b$ this means
  \begin{displaymath}
   \frac{\partial F}{\partial y}-\frac{d}{dx}\frac{\partial F}{\partial \dot y}
   = 0 
  \end{displaymath}
  Which is Lagrange's equation.
  &
\section*{Beltrami Identity}
    Beltrami's Identity is a simplification of Lagrange's equation that can be
    used when $F$ isn't specifically dependent on $x$, $F=F(\{q(x)\}, \{\dot q(x)\})$

    Start with:
    \begin{displaymath}
     \frac{\partial F}{\partial y} = \frac{d}{dx}\frac{\partial F}{\partial \dot
     y} 
    \end{displaymath}
    Multiply by $\dot y$:
    \begin{displaymath}
     \dot y \frac{\partial F}{\partial y}=\dot y \frac{d}{dx}\frac{\partial
       F}{\partial \dot y} 
    \end{displaymath}
    Take the derivative of $F$ wrt. $x$ for subbing in:
    \begin{displaymath}
     \frac{d F}{d x}  = \frac{\partial F}{\partial y}\dot y + \frac{\partial
       F}{\partial \dot y}\ddot y + \frac{\partial F}{\partial x}
    \end{displaymath}
    Rearrange:
    \begin{displaymath}
     \dot y \frac{\partial F}{\partial y}=\frac{dF}{dx}-\frac{\partial
       F}{\partial \dot y}\ddot y - \frac{\partial F}{\partial x} 
    \end{displaymath}
    Sub this in to the left half of the second equation:
    \begin{displaymath}
\frac{dF}{dx}-\frac{\partial
       F}{\partial \dot y}\ddot y - \frac{\partial F}{\partial x} = \dot y
     \frac{d}{dx}\frac{\partial F}{\partial \dot y}
    \end{displaymath}
    The term on the right hand side can be given by:
    \begin{displaymath}
      \dot y \frac{d}{dx}\frac{\partial F}{\partial \dot y} = \frac{d}{dx}\left(
        \dot y \frac{\partial F}{\partial \dot y} \right) - \frac{\partial
        F}{\partial \dot y} \ddot y
    \end{displaymath}
    Subbing this in gives:
    \begin{displaymath}
      \frac{dF}{dx}-\frac{\partial F}{\partial x}= \frac{d}{dx}\left( \dot y
        \frac{\partial F}{\partial \dot y} \right)
    \end{displaymath}
    Since $\frac{\partial F}{\partial x}=0$, we get:
    \begin{displaymath}
     \frac{dF}{dx}-\frac{d}{dx}\left( \dot y \frac{\partial F}{\partial \dot y} \right) =0
    \end{displaymath}
    Which can be integrated to give:
    \begin{displaymath}
     F-\dot y \left (\frac{\partial F}{\partial \dot y} \right ) = C
    \end{displaymath}
    Where $C$ is a constant.
\tabularnewline\hline

\end{longtable}
\end{document}