\documentclass[table,cmyk]{article}
\usepackage[a4paper,margin=0.5cm,landscape]{geometry}
\usepackage{longtable,array,calc}
\usepackage{xcolor}
\usepackage{mathtools}
\usepackage{cancel}
\usepackage{ amssymb }
\usepackage{ulem}
\makeatletter
\newcommand\ratio[2]{\strip@pt\dimexpr#1pt/#2\relax}
\newcommand{\Lagr}{\mathcal{L}}
\newcolumntype{A}[2]
{
        >{\begin{minipage}[t]{#2\linewidth-2\tabcolsep-#1\arrayrulewidth}%
        \vspace{\tabcolsep}}%
        c%
        <{\vspace{\tabcolsep}\end{minipage}}%   
}
\makeatother

\pagestyle{empty}

\arrayrulewidth=1pt
\tabcolsep=10pt
\arrayrulecolor{blue}


\begin{document}
\begin{longtable}
{
    |A{1.5}{\ratio{30}{100}}% 30%
    |A{1}{\ratio{30}{100}}% 30%
    |A{1.5}{\ratio{30}{100}}% 40%
    |%
}\hline
%Cell 1,1
\section*{Newtonian Mechanics}
\fbox{
\begin{minipage}{7.4cm}
\textbf{N1} -
\textit{If no forces act on a body, it remains at rest or moves with constant velocity: $\underline{\dot{v}} = 0$}
\vspace{0.2cm}

\textbf{N2} -  $\underline{\dot{p}} = \underline{F}$
\vspace{0.1cm}

\textbf{N3} - $\underline{F_{ab}} = - \underline{F_{ba}}$

\end{minipage}
}
\[ \underline{L} \equiv \underline{r} \times \underline{p}\]
\[ W_{BA} \equiv \int_{A}^{B} \underline{F}\cdot d\underline{r} = T_B - T_A\]

Orbits (cylindrical polars):

\[\underline{e_r} = \cos\phi\underline{i} + \sin\phi\underline{j}
\]\[
\underline{e_{\phi}} = -\sin\phi\underline{i} + \cos\phi\underline{j}\]
\[\underline{\dot{r}} = \dot{r} \underline{e_r} + r \dot{\phi} \underline{e_{\phi}}\]
\[E = \frac{1}{2}m(\dot{r}^2 + r^2\dot{\phi}^2) +V(r)\]

\noindent\rule{7.8cm}{0.4pt}

\section*{Newton to Lagrange}
Holonomic constraint is an algebraic relation between coordinates:
\[f(\underline{r_a},\underline{r_b},...,\underline{r_N};t) = 0\]
For system with £N cartesian cordinates $x_i$, $M$ constraints, and $3N-M$ generalised coordinates $q_i$, and $x_i = x_i(\{q\},t)$
\newline

Virtual displacement:
\[\delta x_i = \sum_{i} \frac{\partial x_i}{\partial q_j} \delta q_j + 0\]
Generalised forces:
\[Q_j = \sum_{i} F_i \frac{\partial x_i}{\partial q_j}\]
&
for a function $f = f(\{q\},\{\dot{q}\},t)$

\[df = \sum_{j} \frac{\partial f}{\partial q_j} + \sum_{j} \frac {\partial f}{\partial \dot{q_j}} d\dot{q_j} + \frac{\partial f}{\partial t}dt\]

\noindent\rule{7.8cm}{0.4pt}

for a function $f = f(\{q\},\{\cancel{\dot{q}}\},t)$

cancellation of dots

\[ \frac{\partial \dot{f}}{\partial \dot{q}_j} = \frac{\partial f}{\partial q_j}\]

commuting derivatives:

\[\frac{d}{dt}\left(\frac{\partial f}{\partial q_j} \right) = \frac{\partial}{\partial q_j}\left( \frac{df}{dt} \right)\]

\noindent\rule{7.8cm}{0.4pt}

Lagrange's equations (general form):

\[ \frac{d}{dt} \left( \frac{\partial T}{\partial \dot{q}_j} \right) - \frac{\partial T}{\partial q_j} = Q_j\]
or

\vspace{0.1cm}
\fbox{
\begin{minipage}{7.4cm}

\[\frac{d}{dt} \left( \frac{\partial \Lagr}{\partial q_j} \right) - \frac{\partial \Lagr}{\partial q_j} = 0\]

where $\Lagr(\{q\},\{\dot{q}\}.t) = T(\{q\}.\{\dot{q}\},t) - V(\{q\},t)$
\end{minipage}

}

\section*{Calculus of Variations}
Euler's Equation: for functional of the form
\[ I[y(s)] = \int_{a}^{b} F(y(s),y'(s),s)ds\]

\[ \frac{\partial F}{\partial y} - \frac{d}{ds} \left( \frac{\partial F}{\partial y'} \right) = 0 \]

with first integrals

\[\frac{\partial F}{\partial y'} = \text{constant} \text{\hspace{1cm} if F does not depend on y}\]
&
\[y'\frac{\partial F}{\partial y'} - F = \text{constant} \text{\hspace{0.7cm} if F does not depend on s}\]


Hamilton's Principle:
\[\delta S = 0 \text{\hspace{0.3cm}for the \textit{action} \hspace{0.3cm}} S = \int_{t_1}^{t_2} \Lagr dt \]
\subsection*{Energy Function}
  The energy function $(h)$ is usually conserved, whereas the energy $(E=T+V)$
  might not be. $h$ is given by:
  \begin{displaymath}
    h=\sum_i\frac{\partial L}{\partial \dot q_i}\dot q_i-L=\sum_ip_i\dot q_i-L
  \end{displaymath}
  $h$ and $L$ are the legendre transforms of each other on the variables $p_i$
  and $\dot q_i$:
  \begin{displaymath}
    p_i=\frac{\partial L}{\partial \dot q_i}, \dot q_i = \frac{\partial
      h}{\partial p_i}
  \end{displaymath}
  \subsection*{Relativistic Particle}
  The lagrangian for a free relativistic particle can be found by minimising the
  action $S$, then requiring $\frac{\partial L}{\partial \dot q_i}=\gamma m \dot
  q_i$
  \begin{displaymath}
   S=
   \int_{\tau_1}^{\tau_2}\epsilon d\tau=\epsilon\int_{t_1}^{t_2}\frac{dt}{\gamma(\dot
     q)}=\epsilon
  \end{displaymath}
  Finding $\epsilon$ from relativistic momentum:
  \begin{displaymath}
    p_i=\gamma m \dot q_i=\frac{\partial L}{\partial \dot
      q_i}=-\frac{\epsilon\gamma\dot q_i}{c^2}
  \end{displaymath}
  And we have $\epsilon = -mc^2$, so the lagrangian is:
  \begin{displaymath}
   L=-\frac{mc^2}{\gamma} 
  \end{displaymath}
\tabularnewline\hline

\end{longtable}
\end{document}