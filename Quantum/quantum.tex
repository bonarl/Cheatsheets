\documentclass[table,cmyk]{article}
\usepackage[a4paper,margin=0.5cm,landscape]{geometry}
\usepackage{longtable,array,calc}
\usepackage{xcolor}
\usepackage{mathtools}
\usepackage{braket}
\makeatletter
\newcommand\ratio[2]{\strip@pt\dimexpr#1pt/#2\relax}
\newcolumntype{A}[2]
{
        >{\begin{minipage}[t]{#2\linewidth-2\tabcolsep-#1\arrayrulewidth}%
        \vspace{\tabcolsep}}%
        c%
        <{\vspace{\tabcolsep}\end{minipage}}%   
}
\makeatother

\pagestyle{empty}

\arrayrulewidth=1pt
\tabcolsep=10pt
\arrayrulecolor{blue}


\begin{document}
\begin{longtable}
{
    |A{1.5}{\ratio{30}{100}}% 30%
    |A{1}{\ratio{30}{100}}% 30%
    |A{1.5}{\ratio{30}{100}}% 40%
    |%
}\hline
%Cell 1,1
\section*{Basics}
Linear superposition of eigenstates:
\[
\ket{\psi,t} = \sum_{i} c_i(t)\ket{u_i}
\]
\[
\Psi(\vec{r},t) = \sum_{i} c_i(t)u_i(\vec{r})
\]
Probability of getting result:
\[
P(A_i) = |c_i(t)|^2\]
Identity operator in a basis, $i$:
  \[
  \hat I=\sum_{i}\ket{i}\bra{i}
  \]  
Expectation value of observable:
\[\braket{\hat{A}} = \braket{\psi, t|\hat{A}|\psi, t}\]
Uncertainty relations:
\[\Delta\hat{A}_t \equiv (\braket{\hat{A}^2} - \braket{\hat{A}}^2)^{\frac{1}{2}}\]

\[\Delta\hat{A}_t\Delta\hat{B}_t \ge \frac{1}{2} |\braket{[ \hat{A},\hat{B} ]}|\]
Schr\"odinger equation:
  \[
  \hat H\psi = i \hbar \frac{\partial}{\partial t} \psi
  \]
Braket $\leftrightarrow$ Function notation
  \[
  \braket{\psi|\phi}=\int \psi^*\phi\  \mathrm{d}x
  \]




\section*{Angular Momentum \& Spin}
Angular Momentum Operators:
\[
\hat{L^2}\ket{l,m} = l(l+1)\hbar^2\ket{l,m}
\]
&
\[\hat{L^2_z} \ket{l,m} = m\hbar\ket{l,m}\]

Spherical Polars:
\[ \hat{L^2} = -\hbar^2\left[\frac{1}{\sin{\theta}} \frac{\partial}{\partial\theta}\left(\sin{\theta}\frac{\partial}{\partial\theta}\right)+\frac{1}{\sin^2{\theta}} \left( \frac{\partial^2}{\partial\phi^2}\right) \right] \]

\[ \hat{Lz} = -i\hbar\frac{\partial}{\partial\phi}\]

\fbox{
\begin{minipage}{7.4cm}
\vspace{0.3cm}
\subsection*{Quantum Numbers}
\[
l = 0,1,2,3... n
\]
\[m_l = l, l-1,... ,-l\]
\[m_l \textmd{ degeneracy} = (2l+1)\]
\[s = 0, \tfrac{1}{2}, 1, \tfrac{3}{2},...\]
\[m_{s} = s, s-1, ..., -s\]
\[m_s \textmd{ degeneracy} = (2s+1)\]
\[ j = l+s, l+s-1, ..., |l-s+1|, |l-s|\]
Russel Saunders notation labels terms $n^{(2s+1)}l_j$
\vspace{0.1cm}

\end{minipage}
}
\vspace{0.3cm}

Total angular momentum:
\[ \hat{\underline{J}} \equiv \hat{\underline{L}}+\hat{\underline{S}}\]
\[[\hat{J}_i,\hat{J}_j] = i\hbar \varepsilon_{ijk}\hat{J}_k\]
\begin{center}
{\centering

coupled basis $\ket{l, m_l, s, m_s}$ 

uncoupled basis $\ket{l, s, j, m_j}$}
\end{center}{}

Matrix Elements:
\[\braket{s,m'|\hat{S_z}|s,m} = m\hbar \delta_{m',m}\]
\[\braket{s,m'|\hat{S_{\pm}}|s,m} = \sqrt{s(s+1)-m(m\pm1)} \hbar \delta_{m',m\pm1}\]
\subsection*{Variational Method}
\[\langle E\rangle = \frac{\braket{\psi|H|\psi}}{\braket{\psi|psi}} = \frac{\sum_{n }|c_n|^2E_n}{\sum_{n }|c_n|^2} \ge E_1\]
&
\section*{Atoms}


General multi-electron Hamiltonian:
\[\hat{H} = \sum_{i=1}^{N} \Big\{ \frac{\hat{p}_i^2}{2m}-\frac{Ze^2}{(4\pi\epsilon_0)r_i}\Big\} + \sum_{i>j=1}^{N} \frac{e^2}{(4\pi\epsilon_0)r_{ij}}\]

Hydrogen fine structure:
\[\Delta_{nj} = E^{(0)}_n \frac{(Z\alpha)^2}{n^2}\left( \frac{n}{j+\tfrac{1}{2}}-\frac{3}{4}\right)\]
2-Electron wavefunctions:
\[\chi_{1,1} = \alpha_1\alpha_2\]
\[\chi_{1,0} = \tfrac{1}{\sqrt{2}}\{\alpha_1\beta_2+\beta_1\alpha_2\}\]
\[\chi_{1,-1} = \beta_1\beta_2\]
\[\chi_{0,0} = \tfrac{1}{\sqrt{2}}\{\alpha_1\beta_2-\beta_1\alpha_2\}\]
\noindent\rule{7.8cm}{0.4pt}

\section*{Scattering}
Born Approximation
\[\frac{d \sigma}{d \Omega} =\frac{m^2}{4\pi^2\hbar^2}L^6|\braket{\vec{k'}|\hat{V}|\vec{k}}|^2\]
where matrix element $V_{\vec{k'}\vec{k}}$ is given by
\[\braket{\vec{k'}|\hat{V}|\vec{k}} = \frac{1}{L^3}\int V(\vec{r})\exp{(-i\vec{q}\cdot \vec{r})}d^3r\]
for central potentials, Born Approximation becomes
\[\frac{d\sigma}{d\omega} = \frac{4m^2}{\hbar^4K^2}\left|\int_{0}^{\infty} rV(r)sin(Kr)dr\right|^2\]
where $K$ is the wave-vector transfer, $K = 2ksin\frac{\theta}{2}$
\tabularnewline\hline
\section*{\underline{Perturbation Theory}}

\subsection*{Time-Independent Non-Degenerate:}

$\bullet$ let $H = H_o + \lambda H'$

$\bullet$ expand $E_n$ and $\ket{n}$ in orders of lambda

$\bullet$ compare $\lambda$'s

$\bullet$ act with arbitrary vector $\ket{k}$

\[E_n^{(1)} = \braket{n^{(0)}|\hat{H'}|n^{(0)}} \equiv H_{nn}'\]

\[ \ket{n^{(1)}} = \sum_{k\ne n} \frac{H_{kn}'}{(E_n^{(0)}-E_k^{(0)})}\ket{k^{(0)}}\]

\[E_n^{(2)} = \sum_{m\ne n} \frac{|H_{mn}'|^2}{(E_n^{(0)}-E_m^{(0)})}\]

\subsection*{Degeneracy}

$\bullet$ linear superposition of the $g$ degenerate substates is also an eigenstate with the same eigenvalue as substates, i.e
\[\ket{E^{(0)}} = \sum_{n=1}^{g} b_n\ket{E_n^{(0)}}\]
$\bullet$ same procedure as Time-Independent, then take $k \le g$ for the arbitrary state, leads to $g\times g$ matrix which must have $0$ determinant for non-zero solutions
\[ \sum_{n=1}^{g} (H'_{kn} - E^{(1)}\delta_{kn})b_n=0, \text{\hspace{1cm}} k = 1,....,g\]

\[\det(H'_{kn} - E^{(1)}\delta_{kn}) = 0\]

\subsection*{Time-Dependent}
General solution to TDSE
\[\ket{\psi, t} = \sum_{n } c_n^{(0)} e^{-E_n^{(0)}t/\hbar}\ket{n^{(0)}}\]
&
\[i\hbar\frac{\partial}{\partial t}\ket{\psi, t} = H_o \ket{\psi, t}\]

$\bullet$ time-dependent perturbation implies $c_n$ coefficients are time-dependent

$\bullet$ sub general solution into TDSE and take scalar product with general state $\bra{m^{(0)}}$ to obtain differential equation for coefficients

\[\dot{c_m} = (i\hbar)^{-1}\sum_{n }H_{mn}'e^{i\omega_{mn}t}\]

$\bullet$ Expand coefficients and Hamiltonian with $\lambda$ as before, compare orders of $\lambda$

$\bullet$ Integrate differential equation for $\dot{c_m}^{(1)}$

\[c_m^{(1)}(t) = c_m^{(1)}(t_0) + (i\hbar)^{-1}\sum_{n } \int_{t_0}^{t} H_{mn}'e^{i\omega_{mn}t'}dt'\]

$\bullet$ when system is known to initially be in eigenstate k at $t=0$ get transition probability

\[p_{mk}(t) \approx \left|c_m^{(1)}(t)\right|^2 = \frac{1}{\hbar^2}\left|\int_{t_0}^{t}H_{mk}'e^{i\omega_{mn}t'}dt'\right|^2\]
\subsection*{Time-Independent Perturbations in Time-Dependent Perturbation Theory}

\[p_{mk} = \frac{2|H_{mk}'|^2}{\hbar^2} f(t,\omega_{mk})\]

where \[f(t,\omega_{mk}) = \frac{2sin^2(\omega_{mk}t/2)}{\omega_{mk}^2}\]

for constant perturbation, take integral between energy range of transition probability (with energy density) to obtain Fermi's Golden Rule:

\[R = \frac{2\pi}{\hbar} |H_{mk}'|^2 \rho(E)\]

which is transitions per unit time
&
\tabularnewline\hline



\end{longtable}
\end{document}